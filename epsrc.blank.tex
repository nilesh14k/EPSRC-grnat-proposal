%% Old style latex (latex209) form for EPSRC grant proposal evaluation
%% Aaron Sloman 5 Sep 1998
%% Updated to include "research adventure" question: 28 Jun 1999
%% Suggestion for "ticks" from Mike Paterson, Warwick University
%% Suggestions for improvement or correction welcome.
%% Email A.Sloman@cs.bham.ac.uk
%%
%% NB before filling this in you are advised to run it through latex
%% and preview it, to see what you are aiming at. Of course, some
%% sections will expand after you have written comments.
%%

\documentstyle[times]{article}
\pagestyle{empty}
\setlength{\textwidth}{17.5cm}
\setlength{\textheight}{26.5cm}
\setlength{\oddsidemargin}{-0.7cm}
\setlength{\evensidemargin}{-0.7cm}
\setlength{\topmargin}{-1.5cm}
\setlength{\headheight}{0cm}
\setlength{\headsep}{0cm}
\def\RULE{\vspace*{1mm}\hrule\vspace*{2mm}}
%
%% To get a "tick" in a box on page 1, use "\tick"
\def\tick{\raisebox{-1.5ex}{\Large\quad\quad$\surd$}}
%
% different numbers of new lines
\def\NL{\newline}
\def\NLL{\newline \newline}             %not needed
\def\NLLL{\newline \newline \newline}   %not needed
\def\NLLLL{\newline \newline \newline \newline}
%
% vertical space
\def\VS{\vspace*{3mm}}
% Shorter vertical space
\def\vs{\vspace*{1mm}}
%
% Subsection start
\def\SUB{\VS \large \bf}
%
% This is where printable stuff starts
\begin{document}
\parindent 0cm
\parskip 2mm

{\bf
THIS FORM MAY BE PHOTOCOPIED AND PASSED (UNATTRIBUTED) TO THE
INVESTIGATOR}


\begin{tabbing}
%
%% Ignore the next line, used to set spacing
Grant Reference:  \= GR/ \hspace*{4cm} \= Project Title: \hspace*{0.5cm} \= XXXXXX \= \kill
%
%% Fill in first block of XXXX with reference. Second and third with
%% project title (add extra lines if necessary)
Grant Reference: \> GR/XXXX
                        \> Project Title: \> XXXX
\\
                  \>    \>  \> XXXX \\ %%duplicate or delete this line as needed
\\
A1.1 \\
%
%% fill in date (DDDD) and name of proposer (NNNN)
Please return form by: DDDD \> \> Investigator: \> \> NNNN \\
%
%% Name and address of EPSRC official, and Proposer's institution
To: XXXXX  \> \> Institution/Organisation:  \> \> XXXXX \\
IT \& COMPUTER SCIENCE PROGRAMME \> \> \>  \\ %institution can overflow
\>  \> Referee reference number: \> \> XXXX
\end{tabbing}
\RULE

%% See the detailed advice/prompts on official printed form.
{\bf RATING SUMMARY:}  Please ensure that the box markings given below
are consistent with your narrative comments overleaf. EPSRC recommends
that you complete the narrative section overleaf before undertaking the
summary.

%% Rating boxes follow.
%% Percentage under each heading provides a guide to the total
%% number of proposal expected in that category

%% Enter an assessment for each of the criteria by inserting
%% "\tick" after the appropriate "&" in the table.
%% Or do it by hand with a tick on a printed copy.
%
%% Entries in the "confidence" column should use H=high, M=medium or
%% L= low.

{\small
\begin{tabular}{||p{2.0cm}||p{2cm}|p{2cm}|p{2cm}|p{2cm}|p{2cm}||p{1.8cm}||}
\hline
{\normalsize \bf Assessment \NLLLL Criterion}
            & Unsatisfactory \NL
              (20\%)
                & Modest contribution/ ability \NL
                  (30\%)
                    & Nationally  competitive \NL
                      (25\%)
                        & Nationally leading and \NL internationally competitive \NL
                          (20\%)
                            & Internationally leading \NL
                              (5\%)
                                    &  { Confidence level H/M/L} \\
\hline
Scientific \& \NL Technological Quality
%
%% insert \tick after the appropriate one of the first 5 ampersands,
%% and ~~~H or ~~~M or ~~~L after the last one. Or make ticks on paper
%% version after printing.
% Example
%           &   &\tick   &   &   &       &  ~~H \\
%
            &   &   &   &   &       &  \\
\hline
Ability to \NL undertake the \NL research
%
%% insert \tick after the appropriate one of the first 5 ampersands,
%% and ~~~H or ~~~M or ~~~L after the last one. Or write on printed copy.
%
            &   &   &   &   &       & \\
\hline
\end{tabular}
} %end \small

{\small
\begin{tabular}{||p{2cm}||p{2cm}|p{2cm}|p{2cm}|p{2cm}|p{2cm}||p{1.8cm}||}
\hline
        & Unsatisfactory (20\%)
            & Satisfactory (30\%)
                & Good (25\%)
                    & Excellent (20\%)
                        & Outstanding (5\%)
                                &Confidence level H/M/L\\
\hline
Viability \& \NL planning
%% insert \tick after the appropriate one of the first 5 ampersands,
%% and ~~~H or ~~~M or ~~~L after the last one. Or write on printed copy.
        &   &   &   &   &       &  \\
\hline
Relevance to \NL beneficiaries
        &   &   &   &   &       &  \\
\hline
Resources \& \NL cost effectiveness
        &   &   &   &   &       &  \\
\hline
\end{tabular}
} %end \small

%% turning off right-justification from here on seems to improve
%% layout
\raggedright

{\bf
ADDITIONAL COMMENTS:   please  use this space to note any additional
issues e.g. multi-disciplinarity, potential project risks. We would be
interested in any suggestions you may have for improving the project.}
%
%% If desired, increase size of font from here on by uncommenting this:
% \large
%% Then you can fill the space available in fewer words.


\vfill %next bit at bottom of page
{\sc Data
Protection Act: Data from this form will be held on a confidential
computer database.}
\newpage
%% If desired, increase size of font from here on by uncommenting this:
%% \large

{\normalsize\bf
THIS FORM MAY BE PHOTOCOPIED AND PASSED (UNATTRIBUTED) TO THE
INVESTIGATOR\\
\hspace*{2cm}PLEASE PROVIDE A FULL EXPLANATION IN SUPPORT OF YOUR VIEWS}

%% Uncomment some of the prompts if inclusion of their text is desired.
{\SUB
1.  SCIENTIFIC \& TECHNOLOGICAL QUALITY}\\
%% Give your opinion on likely scientific and technological quality
%% of this proposal, i.e. on the nature of the research area, its
%% novelty, the contribution that this research would make and its
%% competitiveness in relation to other work in the field, both in
%% the UK and internationally.


{\SUB
2.  ABILITY  TO UNDERTAKE  THE RESEARCH}\\
%% What are your views on the ability of the investigator/ project team
%% to undertake this research project?}

{\SUB
3.  VIABILITY  AND  PLANNING}\\
%% Please comment on the viability of the proposed methods and
%% techniques and on the suitability of the research infrastructure
%% Please also indicate whether the project objectives, timescales
%% and plans are realistic.

%%delete these sub-headings if unwanted.
{\bf Viability of methods and techniques:}\\

{\bf Suitability of research infrastructure:}\\

{\bf Are objectives realistic:}\\

{\bf Are timescales and plans realistic:}\\

{\SUB
4.  RELEVANCE TO BENEFICIARIES}\\
%% Please comment on the beneficiaries identified by the proposer.
%% How important is this research likely to be to them or to others?


\vs
{\sc Please comment on any specific collaborations and/or on suggestions
for exploitation and dissemination.}

{\SUB
5.  RESOURCES AND COST EFFECTIVENESS}\\
%% Comment on the choice of, and need for, the resources requested,
%% paying particular attention to staff, major capital and facilities.


\vs
{\sc Please give your opinion on the overall cost effectiveness of the
work.}

{\SUB
6.  ADVENTURE IN RESEARCH}\\
%%  EPSRC wishes to encourage ``adventure in research'' subject to
%% the requirement of high quality. You are invited to comment on the
%% degree of adventure in the proposal -- the balance between research
%% risk and potential to make major advances in the field.


\vs
{\sc Provide comments on the element of adventure in the proposal,
and the balance between risk and potential.}


\end{document}
